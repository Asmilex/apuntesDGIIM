\documentclass{article}
\begin{document}
%% Toni dale formato a esto pls

\section{Criterios para polinomios irreducibles}

\title{ Criterio de Eisenstein. }

Este criterio es muy útil pues ofrece una condición suficiente para comprobar que un polinomio sobre los numeros racionales es irreducible.

Si p(x) un polinomio con coeficiente $a_0\cdots a_n$, entonces si existe un número primo $p$ tal que:
\begin{itemize}
     \item $p$ divide a todo a $a_i$ $i \neq n$.
     \item $p$ no divide a $a_n$.
     \item $p^2$ no divide a $a_0$.
\end{itemize}

Entonces p(x) es irreducible.


\title{ Criterio de la raiz. (simplificado)}
En $K[x]$ todo polinomio de grado 1 es irreducible y es asociad a uno de la forma $x-a$ con $x-a/\phi(x) \iff \phi(a)=0$.

De este modo podemos ver esto como una condición necesaria de reducibilidad, pues si $\phi\in K[x]$ tal que $gr(\phi)>1$ tiene raices en $K$ no es irreducible.


\title{ Criterio de Abel. }
Sea $K$ un cuerpo y $K[x]$ su anillo de polinomios, entonces sea $f,g \in K[x]$ con $g$ irreducible y compartiendo una raiz. Entonces toda raiz de $g$ es una raiz de $f$. Además $f = gh$ con $h \in K[x]$
\end{document}
